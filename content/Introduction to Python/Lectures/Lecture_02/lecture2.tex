\documentclass[
    aspectratio=169, 
    usepdftitle=false, 
    xcolor={dvipsnames},
    hyperref={
        colorlinks,
        linkcolor=black,
        urlcolor=blue}
    ]{beamer}
\usetheme{Madrid}
\usepackage{graphicx}
\usepackage{listings}
\usepackage{soul}
\usepackage{amsmath}
\lstset{
    numbers=left,
    xleftmargin=2em,
    frame=single,
    framexleftmargin=0em,
    basicstyle=\footnotesize\ttfamily,
    xleftmargin=.075\textwidth,
    xrightmargin=.075\textwidth,
    numberstyle=\footnotesize\ttfamily,
    upquote=true,
    framesep=10pt,
    numbersep=20pt,
    keywordstyle=\bfseries,
    stringstyle=\textit,
    showstringspaces=false,
    columns=fixed,
    breaklines=true,
}

\lstdefinestyle{output}{
    numbers=none,
    xleftmargin=2em,
    frame=single,
    framexleftmargin=0em,
    basicstyle=\footnotesize\ttfamily,
    xleftmargin=.075\textwidth,
    xrightmargin=.075\textwidth,
    numberstyle=\footnotesize\ttfamily,
    upquote=true,
    framesep=10pt,
    numbersep=20pt,
    keywordstyle=\bfseries,
    stringstyle=\textit,
    showstringspaces=false,
    columns=fixed,
    breaklines=true,
    backgroundcolor=\color{lightgray},
}
% \usepackage{xcolor}

\hypersetup{colorlinks,urlcolor=blue}
\addtobeamertemplate{headline}{\hypersetup{linkcolor=.}}{}
\addtobeamertemplate{footline}{\hypersetup{linkcolor=.}}{}

\definecolor{Light}{gray}{.90}
\sethlcolor{Light}

\let\OldTexttt\texttt
\renewcommand{\texttt}[1]{\OldTexttt{\hl{#1}}}% will affect all \texttt

\title[Introduction to Python]{Introduction to Python}
\subtitle{Lecture 2: Loops and Conditionals}
\author{Daniel Kadyrov}
\date{}

\begin{document}

\begin{frame}
    \titlepage
\end{frame}

\begin{frame}{Loops}
    \begin{itemize}
        \item Loops are used to repeat a block of code a certain number of times.
        \item There are two types of loops in Python: \texttt{for} loops and \texttt{while} loops.
        \begin{itemize}
            \item \texttt{for} loops are used to iterate over a sequence.
            \item \texttt{while} loops are used to repeat a block of code while a condition is true.
        \end{itemize}
        \item Loops are useful for automating repetitive tasks, iterating over data, and more.
        \item Loops can be broken out of using the \texttt{break} keyword.
    \end{itemize}
\end{frame}

\begin{frame}[fragile]{For Loops}
    \begin{itemize}
        \item \texttt{for} loops are used to iterate over a sequence.
        \item The syntax for a \texttt{for} loop is as follows:
        \begin{lstlisting}[language=Python]
for <variable> in <sequence>:
    <code>
        \end{lstlisting}
        \item The \texttt{<variable>} is a variable that will be assigned to each element in the \texttt{<sequence>} one at a time.
        \item The \texttt{<code>} is the code that will be executed for each element in the \texttt{<sequence>}.
        \item The \texttt{<sequence>} can be a list, tuple, string, or any other iterable object.
    \end{itemize}
\end{frame}

\begin{frame}[fragile]{For Loops}
    \begin{itemize}
        \item Here is an example of a \texttt{for} loop:
        \begin{lstlisting}[language=Python]
for i in [1, 2, 3, 4, 5]:
    print(i)
        \end{lstlisting}
        \item This will print the numbers 1 through 5.
        \item The \texttt{<variable>} \texttt{i} is assigned to each element in the list one at a time.
        \item The \texttt{<code>} \texttt{print(i)} is executed for each element in the list.
    \end{itemize}
\end{frame}

\begin{frame}[fragile]{For Loops}
    \begin{itemize}
        \item Here is another example of a \texttt{for} loop:
        \begin{lstlisting}[language=Python]
for i in range(5):
    print(i)
        \end{lstlisting}
        \item This will print the numbers 0 through 4.
        \item The \texttt{range()} function returns a sequence of numbers.
        \item The \texttt{range()} function can take up to three arguments: \texttt{range(start, stop, step)}.
        \item The \texttt{start} argument is the number to start at (default is 0).
        \item The \texttt{stop} argument is the number to stop at (not included).
        \item The \texttt{step} argument is the number to increment by (default is 1).
    \end{itemize}   
\end{frame}
\begin{frame}[fragile]{For Loops}
    \begin{itemize}
        \item Here is another example of a \texttt{for} loop:
        \begin{lstlisting}[language=Python]
for i in range(1, 10, 2):
    print(i)
        \end{lstlisting}
        \item This will print the odd numbers from 1 to 9.
        \item The \texttt{range()} function can be used to iterate over a sequence of numbers.
        \item The \texttt{range()} function can be used to iterate over a sequence of numbers.
    \end{itemize}
\end{frame}
    
\begin{frame}[fragile]{While Loops}
    \begin{itemize}
        \item \texttt{while} loops are used to repeat a block of code while a condition is true.
        \item The syntax for a \texttt{while} loop is as follows:
        \begin{lstlisting}[language=Python]
while <condition>:
    <code>
        \end{lstlisting}
        \item The \texttt{<condition>} is a boolean expression that is evaluated each time the loop is run.
        \item The \texttt{<code>} is the code that will be executed while the \texttt{<condition>} is true.
    \end{itemize}
\end{frame}

\begin{frame}[fragile]{While Loops}
    \begin{itemize}
        \item Here is an example of a \texttt{while} loop:
        \begin{lstlisting}[language=Python]
i = 0
while i < 5:
    print(i)
    i += 1
        \end{lstlisting}
        \item This will print the numbers 0 through 4.
        \item The \texttt{<condition>} \texttt{i < 5} is evaluated each time the loop is run.
        \item The \texttt{<code>} \texttt{print(i)} is executed while the \texttt{<condition>} is true.
        \item The \texttt{<code>} \texttt{i += 1} increments the variable \texttt{i} by 1 each time the loop is run.
    \end{itemize}
\end{frame}

\begin{frame}[fragile]{Nested Loops}
    \begin{itemize}
        \item Loops can be nested inside each other.
        \item Here is an example of a nested \texttt{for} loop:
        \begin{lstlisting}[language=Python]
for i in range(5):
    for j in range(5):
        print(i, j)
        \end{lstlisting}
        \item This will print the numbers 0 through 4.
        \item The \texttt{<code>} \texttt{print(i, j)} is executed for each element in the list.
        \item The \texttt{<variable>} \texttt{i} is assigned to each element in the list one at a time.
        \item The \texttt{<variable>} \texttt{j} is assigned to each element in the list one at a time.
    \end{itemize}
\end{frame}

\begin{frame}{Conditionals}
    \begin{itemize}
        \item Conditionals are used to execute a block of code if a condition is true.
        \item There are three types of conditionals in Python: \texttt{if} statements, \texttt{elif} statements, and \texttt{else} statements.
        \begin{itemize}
            \item \texttt{if} statements are used to execute a block of code if a condition is true.
            \item \texttt{elif} statements are used to execute a block of code if another condition is true.
            \item \texttt{else} statements are used to execute a block of code if no other condition is true.
        \end{itemize}
        \item Conditionals are useful for executing code based on certain conditions.
        \item Conditionals can be nested inside each other.
    \end{itemize}
\end{frame}

\begin{frame}[fragile]{If Statements}
    \begin{itemize}
        \item \texttt{if} statements are used to execute a block of code if a condition is true.
        \item The syntax for an \texttt{if} statement is as follows:
        \begin{lstlisting}[language=Python]
if <condition>:
    <code>
        \end{lstlisting}
        \item The \texttt{<condition>} is a boolean expression that is evaluated.
        \item The \texttt{<code>} is the code that will be executed if the \texttt{<condition>} is true.
    \end{itemize}
\end{frame}

\begin{frame}[fragile]{If Statements}
    \begin{itemize}
        \item Here is an example of an \texttt{if} statement:
        \begin{lstlisting}[language=Python]
if x > 0:
    print("x is positive")
        \end{lstlisting}
        \item This will print \texttt{x is positive} if the variable \texttt{x} is greater than 0.
        \item The \texttt{<condition>} \texttt{x > 0} is evaluated.
        \item The \texttt{<code>} \texttt{print("x is positive")} is executed if the \texttt{<condition>} is true.
    \end{itemize}
\end{frame}

\begin{frame}[fragile]{Elif Statements}
    \begin{itemize}
        \item \texttt{elif} statements are used to execute a block of code if another condition is true.
        \item The syntax for an \texttt{elif} statement is as follows:
        \begin{lstlisting}[language=Python]
if <condition>:
    <code>
elif <condition>:
    <code>
        \end{lstlisting}
        \item The \texttt{<condition>} is a boolean expression that is evaluated.
        \item The \texttt{<code>} is the code that will be executed if the \texttt{<condition>} is true.
    \end{itemize}
\end{frame}

\begin{frame}[fragile]{Elif Statements}
    \begin{itemize}
        \item Here is an example of an \texttt{elif} statement:
        \begin{lstlisting}[language=Python]
if x > 0:
    print("x is positive")
elif x < 0:
    print("x is negative")
        \end{lstlisting}
        \item This will print \texttt{x is positive} if the variable \texttt{x} is greater than 0.
        \item This will print \texttt{x is negative} if the variable \texttt{x} is less than 0.
        \item The \texttt{<condition>} \texttt{x > 0} is evaluated.
        \item The \texttt{<code>} \texttt{print("x is positive")} is executed if the \texttt{<condition>} is true.
        \item The \texttt{<condition>} \texttt{x < 0} is evaluated.
        \item The \texttt{<code>} \texttt{print("x is negative")} is executed if the \texttt{<condition>} is true.
    \end{itemize}
\end{frame}

\begin{frame}[fragile]{Else Statements}
    \begin{itemize}
        \item \texttt{else} statements are used to execute a block of code if no other condition is true.
        \item The syntax for an \texttt{else} statement is as follows:
        \begin{lstlisting}[language=Python]
if <condition>:
    <code>
elif <condition>:
    <code>
else:
    <code>
        \end{lstlisting}
        \item The \texttt{<condition>} is a boolean expression that is evaluated.
        \item The \texttt{<code>} is the code that will be executed if the \texttt{<condition>} is true.
    \end{itemize}   
\end{frame}

\begin{frame}[fragile]{Else Statements}

   Here is an example of an \texttt{else} statement:
        \begin{lstlisting}[language=Python]
if x > 0:
    print("x is positive")
elif x < 0:
    print("x is negative")
else:
    print("x is zero")
        \end{lstlisting}
\end{frame}

\begin{frame}[fragile]{Nested Conditionals}
    \begin{itemize}
        \item Conditionals can be nested inside each other.
        \item This means that conditionals can be inside other conditionals.
        \item Here is an example of nested conditionals:
        \begin{lstlisting}[language=Python]
if x > 0:
    if x > 10:
        print("x is greater than 10")
    else:
        print("x is between 0 and 10")
else:
    print("x is less than or equal to 0")
        \end{lstlisting}
    \end{itemize}
\end{frame}

\begin{frame}[fragile]{Nested Conditionals}
    \begin{itemize}
        \item Here is another example of nested conditionals:
        \begin{lstlisting}[language=Python]
if x > 0:
    if x > 10:
        print("x is greater than 10")
    elif x > 5:
        print("x is between 5 and 10")
    else:
        print("x is between 0 and 5")
else:
    print("x is less than or equal to 0")
        \end{lstlisting}
    \end{itemize}
\end{frame}

% \begin{frame}[fragile]{Nested Conditionals}
%     \begin{itemize}
%         \item This will print \texttt{x is greater than 10} if the variable \texttt{x} is greater than 10.
%         \item This will print \texttt{x is between 5 and 10} if the variable \texttt{x} is between 5 and 10.
%         \item This will print \texttt{x is between 0 and 5} if the variable \texttt{x} is between 0 and 5.
%         \item This will print \texttt{x is less than or equal to 0} if the variable \texttt{x} is less than or equal to 0.
%         \item The \texttt{<condition>} \texttt{x > 0} is evaluated.
%         \item The \texttt{<condition>} \texttt{x > 10} is evaluated if the \texttt{<condition>} \texttt{x > 0} is true.
%         \item The \texttt{<code>} \texttt{print("x is greater than 10")} is executed if the \texttt{<condition>} \texttt{x > 10} is true.
%         \item The \texttt{<condition>} \texttt{x > 5} is evaluated if the \texttt{<condition>} \texttt{x > 0} is true and the \texttt{<condition>} \texttt{x > 10} is false.
%         \item The \texttt{<code>} \texttt{print("x is between 5 and 10")} is executed if the \texttt{<condition>} \texttt{x > 5} is true.
%         \item The \texttt{<code>} \texttt{print("x is between 0 and 5")} is executed if the \texttt{<condition>} \texttt{x > 5} is false.
%         \item The \texttt{<code>} \texttt{print("x is less than or equal to 0")} is executed if the \texttt{<condition>} \texttt{x > 0} is false.
%     \end{itemize}
% \end{frame}

\end{document}